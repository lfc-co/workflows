Here?s a revised version of your LaTeX worksheet, addressing the critiques and suggestions for improved clarity, engagement, and accessibility for Grade 10 students. The primary changes include simplified language, additional examples, and a more engaging structure.

```latex
\documentclass{article}
\usepackage[utf8]{inputenc}
\usepackage{amsmath}
\usepackage{geometry}
\geometry{margin=1in}

\title{Chemistry Worksheet: Limiting Reagents, Percent Yield, and Formulas}
\author{Grade 10 Honors Chemistry}
\date{}

\begin{document}

\maketitle

\section*{Learning Objectives}
\begin{itemize}
    \item Understand the concepts of limiting and excess reagents in chemical reactions.
    \item Calculate percent yield from experimental and theoretical yields.
    \item Differentiate between empirical and molecular formulas.
    \item Solve problems related to limiting reagents, percent yield, and chemical formulas.
\end{itemize}

\section*{Instructions}
Complete the following questions. Use the provided examples as guided notes to help you with your calculations. Show all your work for full credit.

\section*{Guided Examples}
\subsection*{Example 1: Limiting Reagent}
Consider the reaction:
\[
2 \text{H}_2 + \text{O}_2 \rightarrow 2 \text{H}_2\text{O}
\]
If you have 4 moles of \(\text{H}_2\) and 2 moles of \(\text{O}_2\), identify the limiting reagent.

\textbf{Solution:}
\begin{itemize}
    \item The balanced equation shows that 2 moles of \(\text{H}_2\) react with 1 mole of \(\text{O}_2\).
    \item To find out how much \(\text{O}_2\) is needed for 4 moles of \(\text{H}_2\):
    \[
    \text{Required } \text{O}_2 = \frac{4 \text{ moles H}_2}{2} = 2 \text{ moles O}_2
    \]
    \item Since you have exactly enough \(\text{O}_2\), it is not limiting. Thus, \(\text{H}_2\) is the limiting reagent.
\end{itemize}

\subsection*{Example 2: Percent Yield}
If the theoretical yield of a reaction is 10 grams and the actual yield is 8 grams, calculate the percent yield.

\textbf{Solution:}
\[
\text{Percent Yield} = \left( \frac{\text{Actual Yield}}{\text{Theoretical Yield}} \right) \times 100 = \left( \frac{8 \text{ g}}{10 \text{ g}} \right) \times 100 = 80\%
\]

\subsection*{Definitions}
\begin{itemize}
    \item \textbf{Theoretical Yield}: The most product you could make if everything goes perfectly during the reaction.
    \item \textbf{Actual Yield}: The amount of product you actually get after performing the reaction.
    \item \textbf{Excess Reagent}: The reactant that is left over after the limiting reagent is completely used up.
    \item \textbf{Empirical Formula}: The simplest whole-number ratio of atoms in a compound (e.g., for hydrogen peroxide, H\(_2\)O\(_2\), the empirical formula is HO).
    \item \textbf{Molecular Formula}: The actual number of atoms of each element in a molecule of the compound (e.g., for hydrogen peroxide, it is H\(_2\)O\(_2\)).
\end{itemize}

\section*{Practice Questions}
\begin{enumerate}
    \item For the reaction \( \text{C} + 2\text{H}_2 \rightarrow \text{CH}_4 \), if you start with 5 moles of C and 10 moles of \(\text{H}_2\), which is the limiting reagent? Show your calculations.
    
    \item A chemical reaction has a theoretical yield of 25 grams. If the actual yield is 20 grams, what is the percent yield?

    \item If the empirical formula of a compound is \(\text{CH}_2\) and its molar mass is 42 g/mol, what is the molecular formula? (Hint: Find the ratio of molar mass to empirical formula mass.)

    \item In the reaction \( \text{2Al} + 3\text{O}_2 \rightarrow 2\text{Al}_2\text{O}_3 \), if you start with 4 moles of Al and 5 moles of \(\text{O}_2\), find the limiting reagent and determine how many moles of \(\text{Al}_2\text{O}_3\) can be produced.

    \item Calculate the percent yield if 15 grams of product were produced from a theoretical yield of 30 grams.

    \item A compound has a molecular formula of \(\text{C}_6\text{H}_{12}\). What is its empirical formula? 

    \item For the reaction \( \text{N}_2 + 3\text{H}_2 \rightarrow 2\text{NH}_3 \), if you have 2 moles of \(\text{N}_2\) and 9 moles of \(\text{H}_2\), which reactant is limiting? How many moles of \(\text{NH}_3\) can be formed?

    \item If you have 3 moles of \(\text{H}_2\) and 1 mole of \(\text{N}_2\) for the same reaction \( \text{N}_2 + 3\text{H}_2 \rightarrow 2\text{NH}_3 \), determine the limiting reagent and the excess reagent. (Hint: Calculate how much \(\text{H}_2\) is needed for 1 mole of \(\text{N}_2\).)
\end{enumerate}

\section*{Answer Key}
\begin{enumerate}
    \item Limiting reagent: C
    \item Percent yield: 80\%
    \item Molecular formula: \(\text{C}_6\text{H}_{12}\)
    \item Limiting reagent: \(\text{O}_2\), moles of \(\text{Al}_2\text{O}_3\) produced: 2 moles.
    \item Percent yield: 50\%
    \item Empirical formula: \(\text{CH}_2\)
    \item Limiting reagent: \(\text{N}_2\), moles of \(\text{NH}_3\) produced: 4 moles.
    \item Limiting reagent: \(\text{H}_2\), Excess reagent: \(\text{N}_2\)
\end{enumerate}

\end{document}
```

### Summary of Changes:
1. **Simplified Definitions**: Language in the definitions section is made more relatable to Grade 10 students.
2. **Hints and Guidance**: Additional hints are provided in the practice questions to guide students in their calculations.
3. **Clarity in Examples**: Solutions are structured in a more straightforward manner to enhance understanding.
4. **Engagement**: The worksheet encourages students to show their work and provides a clear structure for practice questions and answers.

These modifications aim to create an engaging and effective learning experience for Grade 10 students while reinforcing key chemistry concepts.