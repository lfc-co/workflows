\documentclass[12pt]{article}
\usepackage[margin=0.5in]{geometry}
\usepackage{amsmath}
\usepackage{enumitem}

\begin{document}

\title{Chemistry Worksheet: Limiting Reagents and Yield}
\author{For Eva}
\date{}
\maketitle

\section*{Learning Objectives}
\begin{enumerate}
    \item Understand the concept of limiting and excess reagents in chemical reactions.
    \item Calculate percent yield of a reaction.
    \item Differentiate between empirical and molecular formulas.
    \item Solve for empirical and molecular formulas based on provided data.
\end{enumerate}

\section*{Glossary}
\begin{itemize}
    \item \textbf{Limiting Reagent:} The reactant that is completely consumed in a chemical reaction, limiting the amount of product formed.
    \item \textbf{Excess Reagent:} The reactant that is not completely consumed in a chemical reaction and remains after the reaction has occurred.
    \item \textbf{Percent Yield:} A measure of the efficiency of a reaction, calculated as the ratio of actual yield to theoretical yield, expressed as a percentage.
    \item \textbf{Empirical Formula:} The simplest whole-number ratio of atoms of each element in a compound.
    \item \textbf{Molecular Formula:} The actual number of atoms of each element in a molecule of a compound.
\end{itemize}

\section*{Instructions}
Read the examples below carefully. Then, complete the practice problems that follow. Show all your work, including calculations and reasoning.

\section*{Common Types of Chemical Reactions}
\begin{itemize}
    \item \textbf{Synthesis:} Two or more reactants combine to form a single product.
    \item \textbf{Decomposition:} A single compound breaks down into two or more simpler products.
    \item \textbf{Combustion:} A substance reacts with oxygen, often producing energy in the form of heat and light.
\end{itemize}

\section*{Guided Example 1: Limiting and Excess Reagents}
Consider the reaction:
\[
2 \text{H}_2 + \text{O}_2 \rightarrow 2 \text{H}_2\text{O}
\]
If you start with 4 moles of \(\text{H}_2\) and 2 moles of \(\text{O}_2\):
\begin{itemize}
    \item Calculate the limiting reagent.
    \item Determine the amount of excess reagent left after the reaction.
\end{itemize}

\textbf{Solution:}
\begin{enumerate}
    \item The reaction requires 2 moles of \(\text{H}_2\) for every 1 mole of \(\text{O}_2\).
    \item For 2 moles of \(\text{O}_2\), you need \(2 \times 2 = 4\) moles of \(\text{H}_2\).
    \item Since we have exactly 4 moles of \(\text{H}_2\), \(\text{H}_2\) is not limiting; \(\text{O}_2\) is the limiting reagent.
    \item After the reaction, no \(\text{H}_2\) is left, and there is no excess \(\text{O}_2\) remaining.
\end{enumerate}

\section*{Guided Example 2: Percent Yield}
If 10 grams of \(\text{H}_2\text{O}\) are produced in the above reaction, what is the percent yield if the theoretical yield is 15 grams?

\textbf{Solution:}
\[
\text{Percent Yield} = \left( \frac{\text{Actual Yield}}{\text{Theoretical Yield}} \right) \times 100 = \left( \frac{10 \, \text{g}}{15 \, \text{g}} \right) \times 100 = 66.67\%
\]

\section*{Practice Problems}
\begin{enumerate}
    \item For the reaction \( \text{C}_3\text{H}_8 + 5 \text{O}_2 \rightarrow 3 \text{CO}_2 + 4 \text{H}_2\text{O} \):
    \begin{itemize}
        \item If you start with 2 moles of \(\text{C}_3\text{H}_8\) and 8 moles of \(\text{O}_2\), identify the limiting reagent and calculate how much of the other reactant remains.
    \end{itemize}
    \textbf{Tip:} Write down the mole ratio and calculate how many moles of each reactant are needed.

    \item If the actual yield of water in the previous reaction is 18 grams and the theoretical yield is 25 grams, calculate the percent yield.
    \textbf{Tip:} Use the formula for percent yield provided in the guided example.

    \item Determine the empirical formula of a compound that contains 40\% carbon, 6.67\% hydrogen, and 53.33\% oxygen by mass. Show your calculations.

    \item A compound has an empirical formula of \(\text{CH}_2\) and a molar mass of 56 g/mol. What is its molecular formula? Show how to determine this.

    \item In a reaction where 5 grams of \(\text{A}\) (molar mass = 10 g/mol) reacts with 10 grams of \(\text{B}\) (molar mass = 20 g/mol), which reagent is limiting, and how much of the excess reagent is left? Show steps for your calculations.
\end{enumerate}

\section*{Answer Key}
\begin{enumerate}
    \item Limiting reagent is \(\text{C}_3\text{H}_8\); \(\text{O}_2\) remaining = 5 moles.
    \item Percent yield = 72\%.
    \item Empirical formula = \(\text{C}_3\text{H}_4\text{O}\). (Show calculations for moles).
    \item Molecular formula = \(\text{C}_4\text{H}_8\). (Calculate using molar mass).
    \item Limiting reagent is \(\text{A}\); excess \(\text{B}\) remaining = 5 grams. (Show calculations).
\end{enumerate}

\section*{Common Mistakes and Tips}
\begin{itemize}
    \item Always balance the chemical equation before identifying limiting and excess reagents.
    \item Be careful with unit conversions, especially when moving between grams and moles.
    \item Double-check your calculations for percent yield; mistakes often occur in the division or multiplication steps.
    \item Watch for rounding errors when calculating moles from percentages.
    \item Use dimensional analysis to help with conversions and ensure units match.
\end{itemize}

\end{document}