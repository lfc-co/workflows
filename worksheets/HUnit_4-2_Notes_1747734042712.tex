```latex
\documentclass{article}
\usepackage[utf8]{inputenc}
\usepackage{amsmath}
\usepackage{amsfonts}
\usepackage{geometry}

\title{Worksheet ? HUnit\_4-2\_Notes ? HUnit\_4-2\_Notes}
\author{}
\date{}

\begin{document}

\maketitle

\section*{Schr?dinger's Atom (1926)}
Erwin Schr?dinger, an Austrian physicist, proposed a quantum mechanics model of the atom based on his study of the electron in a hydrogen atom. His model describes atoms with electron cloud regions around the nucleus where electrons are most likely to be found (where the electron cloud is most dense).

\textbf{Key Points:}
\begin{itemize}
    \item The current atomic model incorporates electron clouds instead of defined orbits.
    \item The electron cloud represents a 3D region around the nucleus with the highest probability of finding electrons.
\end{itemize}

\textbf{Practice Problems:}
1. Explain how Schr?dinger's model differs from Bohr's model of the atom.
2. Describe the significance of the electron cloud in terms of electron probability.

\section*{Quantum Numbers}
There are four quantum numbers that describe the properties of electrons in atoms:
\begin{itemize}
    \item Principal quantum number ($n$): indicates the main energy level occupied by an electron.
    \item Angular momentum quantum number ($l$): identifies the shape of sublevels.
    \item Magnetic quantum number ($m_l$): indicates the orientation of orbitals within a sublevel.
    \item Spin quantum number ($m_s$): indicates the spin direction of the electron.
\end{itemize}

\textbf{Practice Problems:}
1. List the four quantum numbers for an electron in the 3rd energy level.
2. Explain Pauli's exclusion principle.

\section*{Quantum Numbers - Detailed Overview}
\subsection*{Principal Quantum Number ($n$)}
- $n = 1, 2, 3, \ldots$
- As $n$ increases, the electron's distance from the nucleus and its energy also increase.

\subsection*{Angular Momentum Quantum Number ($l$)}
- Values: $l = 0$ to $(n-1)$
- Corresponding sublevels are:
    - $l = 0 \rightarrow s$
    - $l = 1 \rightarrow p$
    - $l = 2 \rightarrow d$
    - $l = 3 \rightarrow f$

\subsection*{Magnetic Quantum Number ($m_l$)}
- Values range from $-l$ to $+l$ (inclusive).
- Indicates the orientation of orbitals within a sublevel.

\subsection*{Spin Quantum Number ($m_s$)}
- Can be either $+\frac{1}{2}$ or $-\frac{1}{2}$.
- Indicates the spin direction of an electron.

\textbf{Practice Problems:}
1. For the 3rd energy level, determine the possible values of $l$ and $m_l$.
2. How many orbitals are available in the 3rd energy level?

\section*{Putting It All Together}
To determine the probable location of an electron, we can combine the quantum numbers.

\textbf{Example:}
For the 3rd energy level:
- Quantum number $n = 3$
- Possible $l$ values: $0, 1, 2$
- Corresponding $m_l$ values and total orbitals:
    - $l = 0: m_l = 0$ (1 orbital)
    - $l = 1: m_l = -1, 0, +1$ (3 orbitals)
    - $l = 2: m_l = -2, -1, 0, +1, +2$ (5 orbitals)

Total orbitals: 9 (each orbital can hold 2 electrons).

\textbf{Practice Problems:}
1. Determine the maximum number of electrons in the 3rd energy level.
2. Draw the shapes of the s, p, and d orbitals for the 3rd energy level.

\section*{Self-Quiz}
1. What is the equation that shows the mathematical relationship between wavelength and frequency?
2. What is the equation that shows the mathematical relationship between frequency and energy?

\textbf{Answers to Practice Problems:}
1. Schr?dinger's model describes electrons as existing in clouds rather than fixed orbits, unlike Bohr's model.
2. The Pauli exclusion principle states that no two electrons in an atom can have the same set of four quantum numbers.
3. Possible values for the 3rd energy level: $l = 0, 1, 2$; $m_l$: $0$ for $l=0$; $-1, 0, +1$ for $l=1$; $-2, -1, 0, +1, +2$ for $l=2$.
4. Maximum electrons in 3rd energy level: 18.
5. Shapes of orbitals should be drawn as per standard representations.

\end{document}
```