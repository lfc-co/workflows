```latex
\documentclass{article}
\usepackage{amsmath}
\usepackage{amsfonts}
\usepackage{amssymb}
\usepackage{enumitem}

\title{Worksheet ? HUnit\_4-3\_Notes ? HUnit\_4-3\_Notes}
\author{}
\date{}

\begin{document}

\maketitle

\section*{Topic 1: Introduction to HUnits}
HUnits are a concept used in various fields to measure the effectiveness or performance of a system. Understanding HUnits is crucial for analyzing results and making informed decisions.

\subsection*{Explanatory Notes}
HUnits can be defined based on specific metrics relevant to the system being studied. It is important to define the parameters clearly to ensure accurate measurement.

\subsection*{Practice Problems}
\begin{enumerate}[label=\arabic*.]
    \item Define HUnit in your own words.
    \item List three systems where HUnits can be applied.
\end{enumerate}

\section*{Topic 2: Calculating HUnits}
Calculating HUnits involves using specific formulas that relate to the performance metrics of the system.

\subsection*{Explanatory Notes}
To calculate HUnits, you often need data points such as output, input, and efficiency ratios. The general formula can be represented as:
\[
HUnit = \frac{Output}{Input}
\]

\subsection*{Practice Problems}
\begin{enumerate}[label=\arabic*.]
    \item If a system produces 100 units of output with 50 units of input, calculate the HUnit.
    \item A factory produces 200 items per hour using 80 units of energy. What is the HUnit for this factory?
\end{enumerate}

\section*{Topic 3: Analyzing HUnits}
Analyzing HUnits helps in understanding the performance of a system over time.

\subsection*{Explanatory Notes}
Regular analysis of HUnits can reveal trends and areas for improvement. It is important to compare HUnits across different time periods or systems for effective analysis.

\subsection*{Practice Problems}
\begin{enumerate}[label=\arabic*.]
    \item If the HUnit of a system was 2 last month and is 2.5 this month, what does this indicate about the system's performance?
    \item Compare the HUnits of two different systems and discuss which one is performing better and why.
\end{enumerate}

\section*{Answer Key}
\begin{enumerate}[label=\arabic*.]
    \item HUnit is a measurement of effectiveness or performance in a given system.
    \item Possible systems: Manufacturing, Software Performance, Energy Efficiency.
    \item HUnit = \(\frac{100}{50} = 2\).
    \item HUnit = \(\frac{200}{80} = 2.5\).
    \item An increase from 2 to 2.5 indicates improved performance.
    \item The system with the higher HUnit is performing better; explain the factors contributing to this.
\end{enumerate}

\end{document}
```