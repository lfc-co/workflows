```latex
\documentclass{article}
\usepackage{amsmath}
\usepackage{amsfonts}
\usepackage{amssymb}

\title{Worksheet ? HUnit\_4-4\_Notes ? HUnit\_4-4\_Notes}
\author{}
\date{}

\begin{document}

\maketitle

\section*{Understanding HUnit}
HUnit is a unit testing framework for Haskell. It provides a simple way to write and run tests for your code, ensuring that your functions perform as expected. Understanding the components of HUnit is crucial for effective testing.

\subsection*{Explanatory Notes}
- **Test Cases**: Each test case represents a specific condition or scenario that needs to be checked.
- **Assertions**: Assertions are statements that check if a condition is true. If false, the test fails.
- **Test Suites**: A collection of test cases that can be run together.

\subsection*{Practice Problems}
1. Write a simple HUnit test case for a function that adds two integers.
2. Modify the test case to include assertions that check for both correct and incorrect outputs.

\subsection*{Answer Key}
1. Example test case for addition:
   \begin{verbatim}
   testAddition = TestCase (assertEqual "for (1 + 2)," 3 (1 + 2))
   \end{verbatim}
2. Modified test case:
   \begin{verbatim}
   testAddition = TestCase (do
       assertEqual "for (1 + 2)," 3 (1 + 2)
       assertEqual "for (1 + 2)," 4 (1 + 2) -- This will fail
   )
   \end{verbatim}

\section*{Running HUnit Tests}
Running tests is a critical part of the development process. HUnit provides easy commands to execute your tests and display results.

\subsection*{Explanatory Notes}
- Use the `runTestTT` function to run your tests.
- The results will show how many tests passed and how many failed.

\subsection*{Practice Problems}
1. Write a function that runs the previously defined test case and captures the output.
2. Analyze what the output indicates about the success of your tests.

\subsection*{Answer Key}
1. Example function to run tests:
   \begin{verbatim}
   main = runTestTT (TestList [testAddition])
   \end{verbatim}
2. The output indicates the number of tests that passed and failed, helping you identify issues in your code.

\section*{Best Practices for HUnit Testing}
To ensure that your tests are effective, follow these best practices.

\subsection*{Explanatory Notes}
- **Isolate Tests**: Each test should be independent of others.
- **Clear Assertions**: Use descriptive messages in assertions to clarify failures.
- **Regular Testing**: Run tests frequently during development.

\subsection*{Practice Problems}
1. Create a new test that checks edge cases for a function.
2. Refactor an existing test to improve clarity and independence.

\subsection*{Answer Key}
1. Example edge case test:
   \begin{verbatim}
   testEdgeCase = TestCase (assertEqual "for (0 + 0)," 0 (0 + 0))
   \end{verbatim}
2. Refactored test:
   \begin{verbatim}
   testAdditionRefactored = TestCase (do
       assertEqual "Adding positive numbers" 5 (2 + 3)
       assertEqual "Adding negative numbers" (-5) ((-2) + (-3))
   )
   \end{verbatim}

\end{document}
```