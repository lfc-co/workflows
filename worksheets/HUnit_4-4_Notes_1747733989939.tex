```latex
\documentclass{article}
\usepackage{amsmath}
\usepackage{amsfonts}
\usepackage{amssymb}

\title{Worksheet ? HUnit\_4-4\_Notes ? HUnit\_4-4\_Notes}
\author{}
\date{}

\begin{document}
\maketitle

\section*{Introduction to HUnit}
HUnit is a unit testing framework for the Haskell programming language. It allows developers to write tests for their code and ensure that it behaves as expected. This worksheet will guide you through the basics of HUnit, including how to write tests and interpret results.

\subsection*{Explanatory Notes}
HUnit provides a simple way to create and run tests. A test is typically written as a function that asserts certain properties about your code. For example, if you have a function `add` that adds two numbers, you can create a test to check if `add(2, 3)` equals `5`. 

To use HUnit, you will need to:
1. Import the HUnit module: `import Test.HUnit`
2. Define your tests using the `TestCase` and `TestList` functions.
3. Run the tests with the `runTestTT` function.

\subsection*{Practice Problems}
1. Write a test case for a function `multiply` that multiplies two numbers. The test should check if `multiply(3, 4)` equals `12`.
2. Create a test suite that includes the following tests:
   - `subtract(10, 5)` should equal `5`.
   - `divide(20, 4)` should equal `5`.
3. Run the test suite and check if all tests pass.

\section*{Answer Key}
1. 
\begin{verbatim}
import Test.HUnit

multiplyTest = TestCase (assertEqual "for (multiply 3 4)," 12 (multiply 3 4))
\end{verbatim}

2. 
\begin{verbatim}
import Test.HUnit

testSubtract = TestCase (assertEqual "for (subtract 10 5)," 5 (subtract 10 5))
testDivide = TestCase (assertEqual "for (divide 20 4)," 5 (divide 20 4))

testSuite = TestList [testSubtract, testDivide]
\end{verbatim}

3. 
\begin{verbatim}
main :: IO Counts
main = runTestTT testSuite
\end{verbatim}
When you run this code, it should output a summary of the test results indicating whether all tests passed or if any failed.

\end{document}
```