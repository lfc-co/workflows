```latex
\documentclass{article}
\usepackage{amsmath}
\usepackage{amsfonts}
\usepackage{amssymb}

\title{Worksheet ? HUnit\_4-3\_Notes ? HUnit\_4-3\_Notes}
\author{}
\date{}

\begin{document}
\maketitle

\section*{Introduction to HUnits}
HUnits, or "Heterogeneous Units," are a way to represent quantities that have different units. Understanding how to work with HUnits is essential in various fields including physics, engineering, and economics.

\subsection*{Explanatory Notes}
When dealing with HUnits, it is important to be aware of the basic conversions between units. For instance, 1 meter = 100 centimeters, and 1 kilometer = 1000 meters. Always ensure that you convert to a common unit before performing any calculations.

\subsection*{Practice Problems}
1. Convert 1500 milliliters to liters.
2. If a car travels 60 kilometers in 1 hour, what is its speed in meters per second?
3. You have 5 meters of rope. How many centimeters do you have?

\section*{Unit Conversion Techniques}
Unit conversion involves multiplying the quantity by a conversion factor that equals one.

\subsection*{Explanatory Notes}
A conversion factor is a fraction that represents the relationship between two different units. For example, to convert from inches to centimeters, you can use the conversion factor \( \frac{2.54 \text{ cm}}{1 \text{ inch}} \).

\subsection*{Practice Problems}
1. Convert 10 inches to centimeters.
2. How many grams are in 2.5 kilograms?
3. If a recipe calls for 3 cups of flour, how many milliliters does that equal (1 cup = 236.588 mL)?

\section*{Practical Applications of HUnits}
Understanding HUnits can help in everyday scenarios such as cooking, DIY projects, and travel.

\subsection*{Explanatory Notes}
When cooking, it?s often necessary to convert units to ensure you have the right amount of ingredients. Similarly, when traveling, knowing the distance in different units can help with planning your route.

\subsection*{Practice Problems}
1. A recipe requires 500 grams of sugar. How many kilograms is that?
2. A distance of 2 miles is approximately how many kilometers (1 mile = 1.60934 km)?
3. If a bicycle tire has a diameter of 26 inches, what is that in centimeters?

\section*{Answer Key}
1. 1500 mL = 1.5 L
2. Speed = \( \frac{60 \text{ km}}{1 \text{ hr}} = \frac{60 \times 1000 \text{ m}}{3600 \text{ s}} \approx 16.67 \text{ m/s} \)
3. 5 m = 500 cm

1. 10 inches = \( 10 \times 2.54 = 25.4 \) cm
2. 2.5 kg = 2500 g
3. 3 cups = \( 3 \times 236.588 = 709.764 \) mL

1. 500 g = 0.5 kg
2. 2 miles = \( 2 \times 1.60934 \approx 3.21868 \) km
3. 26 inches = \( 26 \times 2.54 \approx 66.04 \) cm

\end{document}
```