```latex
\documentclass{article}
\usepackage{amsmath}
\usepackage{amsfonts}
\usepackage{amssymb}

\title{Worksheet ? HUnit\_4-3\_Notes ? HUnit\_4-3\_Notes}
\author{}
\date{}

\begin{document}

\maketitle

\section*{Overview of HUnit}
HUnit is a unit testing framework for Haskell that allows developers to create and run tests for their code. It is essential for ensuring code reliability and correctness. This section will cover the basics of setting up HUnit and writing your first test cases.

\subsection*{Explanatory Notes}
To use HUnit, you must first install the framework and import it into your Haskell project. HUnit provides a simple way to define test cases and groups of tests.

\subsection*{Practice Problems}
1. Install HUnit in your Haskell environment and create a simple test file.
2. Write a test case for a function that adds two numbers.
3. Create a test suite that includes multiple test cases.

\section*{Writing Test Cases}
In HUnit, a test case is defined using the `TestCase` type. You can assert conditions using various functions provided by the framework.

\subsection*{Explanatory Notes}
A typical test case might look like this:

\begin{verbatim}
testAddition = TestCase (assertEqual "for (1 + 1)," 2 (1 + 1))
\end{verbatim}

This test checks that the result of adding 1 and 1 equals 2.

\subsection*{Practice Problems}
1. Write a test case for a function that multiplies two numbers.
2. Write a test case that checks if a list is empty.
3. Use assertions to verify the output of a string manipulation function.

\section*{Running Tests}
Once you have written your test cases, you can run them using the HUnit test runner. It will execute all tests and report the results.

\subsection*{Explanatory Notes}
To run your tests, use the following command in your terminal:

\begin{verbatim}
runTestTT tests
\end{verbatim}

This command will execute all defined tests and display the results in the console.

\subsection*{Practice Problems}
1. Create a test suite that includes at least three different test cases.
2. Run your test suite and analyze the output.
3. Modify one of your test cases to ensure it fails, then run the tests again.

\section*{Answer Key}
1. (Answers will vary based on student responses)
2. (Answers will vary based on student responses)
3. (Answers will vary based on student responses)

\end{document}
```