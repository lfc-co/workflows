```latex
\documentclass{article}
\usepackage{amsmath}
\usepackage{amssymb}

\title{Worksheet ? HUnit\_4-4\_Notes ? HUnit\_4-4\_Notes}
\author{}
\date{}

\begin{document}

\maketitle

\section*{Topic 1: Understanding HUnits}
HUnits are units of measurement used to quantify various aspects of a system. They are essential for conducting accurate assessments and evaluations in scientific experiments and theoretical calculations.

\textbf{Explanatory Notes:}
HUnits can represent various quantities, such as energy, force, or distance. It's essential to understand how to convert between different units and apply them correctly in calculations.

\textbf{Practice Problems:}
\begin{enumerate}
    \item Convert 5 HUnits to another measurement unit of your choice.
    \item If one HUnit is equivalent to 10 Joules, what is the energy in Joules for 7 HUnits?
    \item A system exerts a force of 15 HUnits. Convert this force into Newtons if 1 HUnit = 1.5 Newtons.
\end{enumerate}

\section*{Topic 2: Calculating with HUnits}
Calculating with HUnits involves applying mathematical operations to derive meaningful results. 

\textbf{Explanatory Notes:}
When performing calculations, ensure that all units are compatible. This might involve converting units before you can add, subtract, multiply, or divide them.

\textbf{Practice Problems:}
\begin{enumerate}
    \item Add 10 HUnits to 25 HUnits. What is the total?
    \item Subtract 3 HUnits from 20 HUnits.
    \item Multiply 2 HUnits by 4.
\end{enumerate}

\section*{Topic 3: Real-World Applications of HUnits}
HUnits have numerous applications in real-world scenarios, from engineering to physics.

\textbf{Explanatory Notes:}
Understanding how to apply HUnits in practical situations is key to problem-solving in various fields. Consider how these units relate to everyday experiences.

\textbf{Practice Problems:}
\begin{enumerate}
    \item A car travels a distance of 50 HUnits. If it travels at a speed of 10 HUnits/hour, how long does the journey take?
    \item If a machine uses 200 HUnits of energy in one hour, how much energy does it use in 4 hours?
    \item A spring with a force of 12 HUnits is compressed. If the spring constant is 4 HUnits/m, how much is the spring compressed?
\end{enumerate}

\section*{Answer Key}
\begin{enumerate}
    \item 50 HUnits = 50 HUnits (same unit).
    \item 7 HUnits = 70 Joules (7 x 10).
    \item 15 HUnits = 22.5 Newtons (15 x 1.5).
    \item Total = 35 HUnits (10 + 25).
    \item 17 HUnits = 20 - 3 HUnits.
    \item 8 HUnits = 2 x 4.
    \item Time = 5 hours (50 HUnits / 10 HUnits/hour).
    \item Energy = 800 HUnits (200 HUnits x 4).
    \item Compression = 3 m (12 HUnits / 4 HUnits/m).
\end{enumerate}

\end{document}
```