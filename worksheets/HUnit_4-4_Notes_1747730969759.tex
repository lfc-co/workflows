```latex
\documentclass{article}
\usepackage[utf8]{inputenc}
\usepackage{amsmath}
\usepackage{amssymb}

\title{Worksheet ? HUnit_4-4_Notes ? HUnit_4-4_Notes}
\author{}
\date{}

\begin{document}

\maketitle

\section*{Introduction to HUnit}
HUnit is a unit testing framework for the Haskell programming language. It allows developers to write test cases for their code, ensuring that each part of the program works correctly. This section covers the basics of writing and running tests using HUnit.

\subsection*{Explanatory Notes}
HUnit provides a simple way to define test cases, group them, and run them. The main components are:
- Test Cases: Individual tests that check a specific functionality.
- Test Suites: Collections of test cases that can be run together.
- Assertions: Conditions that need to be true for the test to pass.

\subsection*{Practice Problems}
1. Write a simple test case using HUnit that checks if the function `add` correctly adds two numbers.
2. Create a test suite that includes at least three different test cases.
3. Write an assertion that checks if the output of the function `multiply` is correct for the inputs 3 and 4.

\section*{Writing Test Cases}
In HUnit, a test case can be defined using the `TestCase` function. This section will explain how to write effective test cases.

\subsection*{Explanatory Notes}
A test case typically includes:
- A description of what it tests.
- The actual test logic, often using the `assertEqual` function to compare expected and actual results.

\subsection*{Practice Problems}
1. Write a test case that tests the `divide` function for various inputs, including edge cases.
2. Modify the previous test case to include a check for division by zero.

\section*{Running Tests}
Running tests in HUnit is straightforward. You can execute your test suites and see the results immediately.

\subsection*{Explanatory Notes}
To run tests, use the `runTestTT` function, which will execute the test suite and report the results, including how many tests passed and failed.

\subsection*{Practice Problems}
1. Create a main function that runs your test suite and prints the results.
2. Write a test suite that includes tests for all your functions and run it.

\section*{Answer Key}
1. Test case for the `add` function:
   ```haskell
   testAdd = TestCase (assertEqual "for (add 2 3)," 5 (add 2 3))
   ```

2. Test suite example:
   ```haskell
   tests = TestList [testAdd, testSubtract, testMultiply]
   ```

3. Assertion for `multiply` function:
   ```haskell
   assertEqual "for (multiply 3 4)," 12 (multiply 3 4)
   ```

4. Test case for `divide` function:
   ```haskell
   testDivide = TestCase (assertEqual "for (divide 10 2)," 5 (divide 10 2))
   ```

5. Test case for division by zero:
   ```haskell
   testDivideByZero = TestCase (assertError "division by zero" (divide 10 0))
   ```

6. Main function to run tests:
   ```haskell
   main = runTestTT tests
   ```

\end{document}
```