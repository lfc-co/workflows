```latex
\documentclass{article}
\usepackage[utf8]{inputenc}
\usepackage{amsmath}

\title{Worksheet ? HUnit\_4-4\_Notes ? HUnit\_4-4\_Notes}
\author{}
\date{}

\begin{document}

\maketitle

\section*{Topic 1: Introduction to HUnits}
HUnits are a measure used in various fields, such as physics and engineering. They serve as a standard for quantifying certain properties. Understanding HUnits is vital for accurate measurements and calculations.

\subsection*{Explanatory Notes}
- HUnits are units of measurement for specific quantities.
- They are standardized to ensure consistency across calculations.
  
\subsection*{Practice Problems}
1. Convert 5 HUnits to the equivalent in a different measurement system.
2. If 1 HUnit equals 10 meters, how many meters are in 15 HUnits?

\subsection*{Answer Key}
1. (Provide relevant conversion details depending on the other measurement system).
2. 150 meters.

\section*{Topic 2: Applying HUnits in Calculations}
HUnits can be integrated into equations to solve real-world problems.

\subsection*{Explanatory Notes}
- HUnits must be consistent across all factors in a calculation.
- Always double-check unit conversions before finalizing answers.

\subsection*{Practice Problems}
1. Calculate the total distance traveled in HUnits if an object moves 20 meters in 2 seconds.
2. If a car travels at a speed of 60 HUnits per hour, how far will it go in 1.5 hours?

\subsection*{Answer Key}
1. 20 HUnits (if 1 HUnit = 1 meter).
2. 90 HUnits.

\section*{Topic 3: Advanced Applications of HUnits}
In advanced applications, HUnits can be used in complex formulas involving multiple variables.

\subsection*{Explanatory Notes}
- Familiarity with algebraic manipulation is essential.
- Units must be maintained throughout the calculation for accuracy.

\subsection*{Practice Problems}
1. If the formula for force is \( F = m \cdot a \) where \( m \) is mass in HUnits and \( a \) is acceleration in HUnits, calculate the force when \( m = 10 \) HUnits and \( a = 2 \) HUnits.
2. Given that energy is calculated using \( E = F \cdot d \), find the energy if \( F = 20 \) HUnits and \( d = 5 \) HUnits.

\subsection*{Answer Key}
1. \( F = 20 \) HUnits.
2. \( E = 100 \) HUnits.

\end{document}
```