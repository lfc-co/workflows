```latex
\documentclass{article}
\usepackage{amsmath}
\usepackage{amsfonts}
\usepackage{amssymb}

\title{Worksheet ? HUnit\_4-4\_Notes ? HUnit\_4-4\_Notes}
\author{}
\date{}

\begin{document}

\maketitle

\section*{Topic 1: Understanding Units of Measurement}
In this section, we will explore various units of measurement, including length, mass, and volume. It is important to understand how to convert between different units and the significance of using the correct unit in calculations.

\textbf{Practice Problems:}
\begin{enumerate}
    \item Convert 5 kilometers to meters.
    \item If a recipe calls for 200 grams of flour, how many kilograms is that?
    \item Convert 3 liters to milliliters.
\end{enumerate}

\textbf{Answer Key:}
\begin{enumerate}
    \item 5 kilometers = 5000 meters.
    \item 200 grams = 0.2 kilograms.
    \item 3 liters = 3000 milliliters.
\end{enumerate}

\section*{Topic 2: Applying Unit Conversions in Real-Life Scenarios}
This section covers how to apply unit conversions in everyday situations, such as cooking, traveling, and scientific experiments. Understanding these conversions can help you make accurate measurements.

\textbf{Practice Problems:}
\begin{enumerate}
    \item You are driving at a speed of 90 kilometers per hour. How many meters do you travel in 30 minutes?
    \item A container holds 1.5 gallons of water. Convert this volume into liters (1 gallon = 3.78541 liters).
    \item A recipe requires 250 milliliters of milk. How many cups is this? (1 cup = 236.588 milliliters)
\end{enumerate}

\textbf{Answer Key:}
\begin{enumerate}
    \item 90 km/h = 25 m/s, so in 30 minutes (1800 seconds), you travel 45000 meters.
    \item 1.5 gallons = 5.678 liters.
    \item 250 milliliters ? 1.06 cups.
\end{enumerate}

\section*{Topic 3: Advanced Unit Conversion Techniques}
In this section, we delve into more complex unit conversions, including converting between non-standard units and using dimensional analysis.

\textbf{Practice Problems:}
\begin{enumerate}
    \item If the speed of light is approximately \(3 \times 10^8\) meters per second, convert this speed into kilometers per hour.
    \item A car's fuel efficiency is rated at 25 miles per gallon. Convert this to kilometers per liter (1 mile = 1.60934 kilometers, 1 gallon = 3.78541 liters).
    \item How many joules are in 500 calories? (1 calorie = 4.184 joules)
\end{enumerate}

\textbf{Answer Key:}
\begin{enumerate}
    \item Speed of light = \(3 \times 10^8 \times \frac{3600}{1000} = 1.08 \times 10^9\) kilometers per hour.
    \item 25 miles/gallon = 10.63 km/L.
    \item 500 calories = 2092 joules.
\end{enumerate}

\end{document}
```