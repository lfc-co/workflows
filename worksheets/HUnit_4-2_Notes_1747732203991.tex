```latex
\documentclass{article}
\usepackage{amsmath}
\usepackage{amssymb}
\usepackage{enumitem}

\title{Worksheet ? HUnit\_4-2\_Notes ? HUnit\_4-2\_Notes}
\author{}
\date{}

\begin{document}

\maketitle

\section*{Understanding HUnits}
HUnits are standardized units used to measure various physical quantities. In this section, we will explore what HUnits are, their significance, and how to convert between different units.

\subsection*{Explanatory Notes}
HUnits serve as the foundation for measuring dimensions such as length, mass, and time. Understanding how to utilize HUnits is essential for solving problems in physics and engineering.

\subsection*{Practice Problems}
1. Convert 15 HUnits of length into meters, knowing that 1 HUnit = 0.5 meters.
2. If a mass of 30 HUnits is equivalent to how many kilograms, given that 1 HUnit = 2 kg?

\section*{Calculating with HUnits}
In this section, we will focus on how to perform calculations involving HUnits.

\subsection*{Explanatory Notes}
Calculations with HUnits often involve addition, subtraction, multiplication, and division. It is crucial to ensure that units are consistent throughout the calculations.

\subsection*{Practice Problems}
1. If an object weighs 10 HUnits and another weighs 20 HUnits, what is their total weight in HUnits?
2. A distance of 40 HUnits is traveled in 5 seconds. What is the speed in HUnits per second?

\section*{Applications of HUnits}
HUnits can be applied in various fields, including physics, engineering, and everyday life. 

\subsection*{Explanatory Notes}
Understanding how to apply HUnits in real-world situations is vital. This section will cover examples of HUnits in action.

\subsection*{Practice Problems}
1. An engineer designs a beam that is 50 HUnits long. If the beam needs to support a weight of 100 HUnits, how much weight can it safely hold?
2. A car travels 100 HUnits in 10 seconds. Calculate its average speed.

\section*{Answer Key}
\subsection*{Understanding HUnits}
1. 15 HUnits = 15 * 0.5 = 7.5 meters.
2. 30 HUnits = 30 * 2 = 60 kg.

\subsection*{Calculating with HUnits}
1. Total weight = 10 + 20 = 30 HUnits.
2. Speed = 40 HUnits / 5 seconds = 8 HUnits/sec.

\subsection*{Applications of HUnits}
1. The beam can safely hold 100 HUnits of weight.
2. Average speed = 100 HUnits / 10 seconds = 10 HUnits/sec.

\end{document}
```