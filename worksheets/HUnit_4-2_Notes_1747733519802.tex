```latex
\documentclass{article}
\usepackage{amsmath}
\usepackage{amssymb}
\usepackage{enumitem}

\title{Worksheet ? HUnit\_4-2\_Notes ? HUnit\_4-2\_Notes}
\author{}
\date{}

\begin{document}

\maketitle

\section*{Introduction to HUnit}
HUnit is a unit testing framework for Haskell that allows developers to write tests for their programs. This section provides a brief overview of how to set up and use HUnit.

\textbf{Key Concepts:}
\begin{itemize}
    \item \textbf{Test Cases:} Individual tests that check a specific aspect of your code.
    \item \textbf{Test Suites:} A collection of test cases that can be run together.
    \item \textbf{Assertions:} Methods to check if a certain condition holds true.
\end{itemize}

\textbf{Practice Problems:}
1. Write a simple test case that checks if a function `add` correctly adds two numbers.
2. Create a test suite that includes multiple test cases.

\section*{Setting Up HUnit}
To use HUnit, you need to install the HUnit library. You can do this using the following command:

\begin{verbatim}
cabal install HUnit
\end{verbatim}

Once installed, you can import the library in your Haskell file:

\begin{verbatim}
import Test.HUnit
\end{verbatim}

\textbf{Practice Problems:}
1. Describe how to set up a new Haskell project to include HUnit.
2. Write a Haskell function and its corresponding test case.

\section*{Writing Test Cases}
A test case is written using the `TestCase` function. Here is a simple example:

\begin{verbatim}
test1 = TestCase (assertEqual "for (add 1 2)," 3 (add 1 2))
\end{verbatim}

\textbf{Practice Problems:}
1. Write a test case for a function `multiply` that multiplies two numbers.
2. What will happen if the assertion fails?

\section*{Running Test Suites}
Test suites can be created using the `TestList` function, and they can be executed using the `runTestTT` function. Here?s an example:

\begin{verbatim}
tests = TestList [test1, test2]
runTestTT tests
\end{verbatim}

\textbf{Practice Problems:}
1. Create a test suite that includes at least three different test cases.
2. Explain how to interpret the output of `runTestTT`.

\section*{Answer Key}
\textbf{Practice Problems Answers:}

1. For the test case checking the `add` function:
   - Define `add` and create a test case like `testAdd = TestCase (assertEqual "add test" 3 (add 1 2))`.
  
2. Setting up a new Haskell project:
   - Create a new directory and run `cabal init`.
   - Add `HUnit` to the dependencies in the `.cabal` file.
  
3. Creating a test case for `multiply`:
   - `testMultiply = TestCase (assertEqual "multiply test" 6 (multiply 2 3))`.

4. If an assertion fails, HUnit will report it in the test results, indicating which test failed and why.

5. Test suite example:
   - `tests = TestList [testAdd, testMultiply]`.

6. Interpreting `runTestTT` output:
   - It indicates the number of tests run, how many passed, and how many failed.

\end{document}
```