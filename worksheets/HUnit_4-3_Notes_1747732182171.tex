```latex
\documentclass{article}
\usepackage{amsmath}
\usepackage{amssymb}
\usepackage{enumitem}

\title{Worksheet ? HUnit\_4-3\_Notes ? HUnit\_4-3\_Notes}
\author{}
\date{}

\begin{document}

\maketitle

\section*{Topic 1: Understanding HUnits}
HUnits (Hypothetical Units) are a way to measure quantities in a simplified manner. They can be used in various contexts such as physics, economics, and engineering. HUnits help us to standardize measurements and make comparisons easier.

\textbf{Practice Problems:}
\begin{enumerate}[label=\arabic*.]
    \item If 1 HUnit is defined as 5 meters, how many meters are there in 3 HUnits?
    \item If a car travels at a speed of 60 HUnits per hour, how far will it travel in 2 hours?
    \item Convert 15 HUnits to meters if 1 HUnit = 4.5 meters.
\end{enumerate}

\textbf{Answer Key:}
\begin{enumerate}[label=\arabic*.]
    \item 15 meters
    \item 120 HUnits (or 600 meters)
    \item 67.5 meters
\end{enumerate}

\section*{Topic 2: Applications of HUnits}
HUnits can be applied in various fields. For example, in physics, they can represent forces, distances, or energy. Understanding how to apply these units is crucial for solving real-world problems.

\textbf{Practice Problems:}
\begin{enumerate}[label=\arabic*.]
    \item A box weighs 10 HUnits. If one HUnit is equivalent to 2 kg, what is the weight of the box in kilograms?
    \item If a recipe requires 3 HUnits of sugar, how much sugar is needed in grams if 1 HUnit = 100 grams?
    \item An object is dropped from a height of 12 HUnits. How many meters did it fall if 1 HUnit = 1.5 meters?
\end{enumerate}

\textbf{Answer Key:}
\begin{enumerate}[label=\arabic*.]
    \item 20 kg
    \item 300 grams
    \item 18 meters
\end{enumerate}

\section*{Topic 3: Converting HUnits}
Converting between HUnits and other measurement systems is an essential skill. Knowing how to convert properly ensures accuracy in calculations and comparisons.

\textbf{Practice Problems:}
\begin{enumerate}[label=\arabic*.]
    \item Convert 25 HUnits to feet if 1 HUnit = 3.28084 feet.
    \item If 10 HUnits = 8 liters, how many liters are in 1 HUnit?
    \item A tank holds 50 HUnits of water. How many gallons is that if 1 HUnit = 0.264172 gallons?
\end{enumerate}

\textbf{Answer Key:}
\begin{enumerate}[label=\arabic*.]
    \item 82.021 feet
    \item 0.8 liters
    \item 13.2086 gallons
\end{enumerate}

\end{document}
```