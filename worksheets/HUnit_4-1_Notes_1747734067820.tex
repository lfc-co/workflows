```latex
\documentclass{article}
\usepackage[utf8]{inputenc}
\usepackage{amsmath}
\usepackage{amssymb}

\title{Worksheet ? HUnit\_4-1\_Notes ? HUnit\_4-1\_Notes}
\author{}
\date{}

\begin{document}

\maketitle

\section*{The Periodic Table}
The periodic table is a tabular arrangement of the chemical elements, organized by increasing atomic number. It reflects the periodic trends in the properties of the elements. The contributions of Dmitri Mendeleev and Henry Moseley are foundational to our current understanding of the periodic table.

\subsection*{Mendeleev's Contributions}
In 1869, Mendeleev organized the known elements by increasing atomic mass, grouping elements with similar properties together. His work allowed for the prediction of properties of undiscovered elements. There were some missing elements in his table, but by 1886, elements such as Scandium (Sc), Gallium (Ga), and Germanium (Ge) were discovered, confirming his predictions.

\subsection*{Moseley's Contributions}
In 1913, Henry Moseley reorganized the elements by increasing atomic number, resolving issues present in Mendeleev's arrangement. This is the organization used in the periodic table today.

\section*{Reading the Periodic Table}
The periodic table consists of 18 vertical columns (groups) and 7 horizontal rows (periods). Elements can be classified as metals, nonmetals, or metalloids based on their position in the table.

\subsection*{Classification}
- **Metals:** Typically found on the left side and in the center of the periodic table.
- **Nonmetals:** Located on the right side of the periodic table.
- **Metalloids:** Found along the zigzag line that divides metals and nonmetals.

\section*{Practice Problems}
1. Name the two ways elements were organized by Mendeleev.
2. How did Moseley organize the known elements at the time?
3. How is the current periodic table organized?
4. Name a metalloid in period 3. What is its symbol?
5. What group is fluorine in and what is the special group name?
6. What group is Argon (Ar) in and what is the special group name?
7. What group is sodium in and what is the special group name?
8. What group is calcium in and what is the special group name?
9. What is another name for Group A elements?
10. What is another name for Group B elements?
11. What period and group is nitrogen in?
12. Name a nonmetal in group 6A in energy level 2. What is its symbol?
13. Name a transition metal in group 7B with n = 4. What is its symbol?
14. What orbital group (s, p, d, or f) is potassium (K) in?
15. What orbital group is iron (Fe) in?

\section*{Answers to Practice}
1. By increasing atomic mass and similar properties.
2. By increasing atomic number.
3. By increasing atomic number.
4. Silicon (Si).
5. 7A, Halogen.
6. 8A, Noble Gas.
7. 1A, Alkali Metal.
8. 2A, Alkaline Earth Metal.
9. Representative.
10. Transition Metals.
11. Period 2, Group 5A.
12. Oxygen (O).
13. Manganese (Mn).
14. s.
15. d.

\end{document}
```