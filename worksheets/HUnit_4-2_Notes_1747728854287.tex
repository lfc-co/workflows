```latex
\documentclass{article}
\usepackage{amsmath}
\usepackage{amssymb}

\title{Worksheet ? HUnit\_4-2\_Notes ? HUnit\_4-2\_Notes}
\author{}
\date{}

\begin{document}

\maketitle

\section*{Topic 1: Introduction to HUnit}
HUnit is a unit testing framework for Haskell. It allows developers to write tests for their code to ensure that it behaves as expected.

\subsection*{Explanatory Notes}
Unit testing is vital in software development as it helps catch bugs early. HUnit provides functions to define and run tests, making it easier to validate the logic of Haskell programs.

\subsection*{Practice Problems}
1. Write a simple HUnit test for a function that adds two numbers.
2. Create a test case that checks if a list is empty.

\subsection*{Answer Key}
1. 
\begin{verbatim}
import Test.HUnit

add :: Int -> Int -> Int
add x y = x + y

testAdd = TestCase (assertEqual "for (add 2 3)," 5 (add 2 3))
\end{verbatim}

2. 
\begin{verbatim}
isListEmpty :: [a] -> Bool
isListEmpty lst = null lst

testIsListEmpty = TestCase (assertEqual "for (isListEmpty [])," True (isListEmpty []))
\end{verbatim}

\section*{Topic 2: Writing Test Cases}
Writing effective test cases is crucial for good unit testing. Each test case should ideally test one specific behavior.

\subsection*{Explanatory Notes}
A test case consists of a description, the actual test, and the expected outcome. This structure helps in understanding what each test is validating.

\subsection*{Practice Problems}
1. Create a test case for a function that checks if a number is even.
2. Write a test for a function that reverses a string.

\subsection*{Answer Key}
1. 
\begin{verbatim}
isEven :: Int -> Bool
isEven n = n `mod` 2 == 0

testIsEven = TestCase (assertEqual "for (isEven 4)," True (isEven 4))
\end{verbatim}

2. 
\begin{verbatim}
reverseString :: String -> String
reverseString str = reverse str

testReverseString = TestCase (assertEqual "for (reverseString \"hello\")," "olleh" (reverseString "hello"))
\end{verbatim}

\section*{Topic 3: Running Tests}
After writing tests, the next step is to run them and check the results.

\subsection*{Explanatory Notes}
HUnit provides a convenient way to run all tests at once. You can use the `runTestTT` function to execute the tests and see which ones pass or fail.

\subsection*{Practice Problems}
1. Write a main function to run all the tests you have created.
2. Describe what the output indicates when a test fails.

\subsection*{Answer Key}
1. 
\begin{verbatim}
main :: IO Counts
main = runTestTT $ TestList [testAdd, testIsListEmpty, testIsEven, testReverseString]
\end{verbatim}

2. If a test fails, it indicates that the actual output did not match the expected output, meaning there might be a bug in the code being tested.

\end{document}
```