```latex
\documentclass{article}
\usepackage[utf8]{inputenc}
\usepackage{amsmath}

\title{Worksheet ? HUnit\_4-2\_Notes ? HUnit\_4-2\_Notes}
\author{}
\date{}

\begin{document}

\maketitle

\section*{Topic 1: Understanding HUnits}
HUnits are units of measurement used to quantify various aspects in mathematics and science. It is important to understand how to convert between different units and how to apply these conversions in problem-solving.

\subsection*{Explanatory Notes}
- HUnits can represent different quantities, such as length, mass, or time.
- Conversions between HUnits and other units must be done using conversion factors.
- Always ensure that the units are consistent throughout your calculations.

\subsection*{Practice Problems}
1. Convert 5 HUnits to Standard Units.
2. If 1 HUnit = 2 Standard Units, how many Standard Units are in 10 HUnits?
3. A measurement of 15 HUnits needs to be converted to Standard Units. What is the result?

\section*{Topic 2: Application of HUnits in Real-World Problems}
Understanding how to apply HUnits in real-world contexts is crucial for solving practical problems.

\subsection*{Explanatory Notes}
- HUnits can be applied in various fields, including physics, engineering, and everyday life.
- Knowing how to apply these units correctly can help in measurements, calculations, and predictions.

\subsection*{Practice Problems}
1. A car travels 60 HUnits in 2 hours. What is its speed in HUnits per hour?
2. If a recipe requires 4 HUnits of flour and you have 10 HUnits, how much flour do you have left after making the recipe?
3. An object weighs 8 HUnits. What is its weight in Standard Units if 1 HUnit = 3 Standard Units?

\section*{Answer Key}
\subsection*{Topic 1 Answers}
1. 5 HUnits = 10 Standard Units.
2. 10 HUnits = 20 Standard Units.
3. 15 HUnits = 30 Standard Units.

\subsection*{Topic 2 Answers}
1. Speed = 30 HUnits per hour.
2. Flour left = 10 HUnits - 4 HUnits = 6 HUnits.
3. Weight = 24 Standard Units.

\end{document}
```