```latex
\documentclass{article}
\usepackage[utf8]{inputenc}
\usepackage{amsmath}
\usepackage{amsfonts}
\usepackage{amssymb}

\title{Worksheet ? HUnit_4-3_Notes ? HUnit_4-3_Notes}
\author{}
\date{}

\begin{document}

\maketitle

\section*{Understanding HUnit}
HUnit is a unit testing framework for Haskell, designed to support the development of reliable and maintainable code. It allows developers to write tests for their functions and check that they behave as expected.

\subsection*{Explanatory Notes}
Unit tests are small tests that verify the functionality of a particular section of code, typically a function or method. HUnit provides a simple way to define and run these tests in Haskell.

Key concepts:
- \textbf{Assertions}: These are statements that check if a condition is true. If the condition is false, the test fails.
- \textbf{Test Suites}: A collection of tests that can be run together.
- \textbf{Test Cases}: Individual tests that check specific functionality.

\subsection*{Practice Problems}
1. Write a simple HUnit test that checks if the function `add` correctly sums two integers.
2. Create a test suite that includes at least three different test cases for a function that calculates the factorial of a number.
3. Write assertions to verify the output of a function that reverses a string.

\section*{Example Code Snippet}
Here is an example of how to set up a basic HUnit test:

\begin{verbatim}
import Test.HUnit

-- Function to be tested
add :: Int -> Int -> Int
add x y = x + y

-- Test case
testAdd = TestCase (assertEqual "for (add 1 2)," 3 (add 1 2))

-- Test suite
tests = TestList [testAdd]

main :: IO Counts
main = runTestTT tests
\end{verbatim}

\subsection*{Answer Key}
1. Example test for the `add` function:
   ```haskell
   testAdd = TestCase (assertEqual "for (add 1 2)," 3 (add 1 2))
   ```
   This test checks if adding 1 and 2 returns 3.

2. Example test suite for the factorial function:
   ```haskell
   testFactorial1 = TestCase (assertEqual "factorial of 0" 1 (factorial 0))
   testFactorial2 = TestCase (assertEqual "factorial of 1" 1 (factorial 1))
   testFactorial3 = TestCase (assertEqual "factorial of 5" 120 (factorial 5))
   ```
   This suite includes three tests for the `factorial` function.

3. Example assertions for the string reversal function:
   ```haskell
   assertEqual "for (reverse 'abc')," "cba" (reverse "abc")
   assertEqual "for (reverse 'hello')," "olleh" (reverse "hello")
   ```

\end{document}
```