```latex
\documentclass{article}
\usepackage{amsmath}
\usepackage{amssymb}
\usepackage{enumitem}

\title{Worksheet ? HUnit\_4-2\_Notes ? HUnit\_4-2\_Notes}
\author{}
\date{}

\begin{document}

\maketitle

\section*{Understanding HUnit}
HUnit is a unit testing framework for Haskell that allows developers to write tests for their code in a structured way. It helps in ensuring that individual components of a program function correctly.

\textbf{Key Concepts:}
\begin{itemize}
    \item \textbf{Test Cases:} Individual tests that check specific functionality.
    \item \textbf{Test Suites:} A collection of test cases that can be run together.
    \item \textbf{Assertions:} Statements that check if a condition is true.
\end{itemize}

\textbf{Practice Problems:}
\begin{enumerate}
    \item Write a simple test case using HUnit to check if the addition function works correctly. 
    \item Create a test suite that includes at least three different test cases.
\end{enumerate}

\section*{Writing Test Cases}
Writing effective test cases is crucial for ensuring software quality. Each test case should be clear, concise, and cover a specific behavior of the code.

\textbf{Format of a Test Case:}
\begin{verbatim}
testName = TestCase (assertEqual "description" expected actual)
\end{verbatim}

\textbf{Practice Problems:}
\begin{enumerate}
    \item Write a test case for a function that calculates the maximum of two numbers.
    \item Modify the previous test case to include a scenario where both numbers are equal.
\end{enumerate}

\section*{Running Tests}
Once test cases are written, they can be executed to check if the code behaves as expected. 

\textbf{Command to Run Tests:}
\begin{verbatim}
runTestTT tests
\end{verbatim}

\textbf{Practice Problems:}
\begin{enumerate}
    \item Write the command to run a test suite named `myTests`.
    \item Explain what happens when a test fails.
\end{enumerate}

\section*{Answer Key}
\textbf{Answers to Practice Problems:}

\section*{Understanding HUnit}
1. Example test case:
\begin{verbatim}
testAddition = TestCase (assertEqual "1 + 1 should equal 2" 2 (1 + 1))
\end{verbatim}
2. Example test suite:
\begin{verbatim}
myTests = TestList [testAddition, testSubtraction, testMultiplication]
\end{verbatim}

\section*{Writing Test Cases}
1. Example test case for max function:
\begin{verbatim}
testMax = TestCase (assertEqual "max of 3 and 5 should be 5" 5 (max 3 5))
\end{verbatim}
2. Modified test case:
\begin{verbatim}
testMaxEqual = TestCase (assertEqual "max of 5 and 5 should be 5" 5 (max 5 5))
\end{verbatim}

\section*{Running Tests}
1. Command to run `myTests`:
\begin{verbatim}
runTestTT myTests
\end{verbatim}
2. When a test fails, it indicates that the actual output did not match the expected output, which helps in identifying issues in the code.

\end{document}
```