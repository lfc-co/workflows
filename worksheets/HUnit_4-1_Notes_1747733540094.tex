```latex
\documentclass{article}
\usepackage{amsmath}
\usepackage{amssymb}
\usepackage{geometry}
\geometry{margin=1in}

\title{Worksheet ? HUnit\_4-1\_Notes ? HUnit\_4-1\_Notes}
\author{}
\date{}

\begin{document}

\maketitle

\section*{Introduction to HUnit}
HUnit is a unit testing framework for the Haskell programming language. It allows developers to write tests for their Haskell code, ensuring that the functions behave as expected. Unit testing is an essential part of software development, as it helps catch bugs early and supports code refactoring.

\subsection*{Practice Problems}
1. Write a simple function in Haskell that adds two numbers. Then, create a unit test using HUnit to verify that the function returns the correct result for the input values 2 and 3.
2. Modify the previous function to handle negative numbers. Write at least two unit tests: one that tests the addition of a negative number and another that tests the addition of two negative numbers.

\section*{Creating Tests with HUnit}
To use HUnit, you need to import the HUnit library in your Haskell file. You can create test cases using the `TestCase` function and group them with `TestList`.

\subsection*{Practice Problems}
1. Create a test suite that contains three test cases for a function that multiplies two integers. Ensure you cover edge cases such as multiplying by zero.
2. Write a test for a function that checks if a number is prime. Include tests for both prime and non-prime numbers.

\section*{Running Tests}
You can run your HUnit tests from the command line using the GHCi interpreter or by compiling your Haskell file and executing it. Make sure to check the output to see if any of your tests fail.

\subsection*{Practice Problems}
1. Write a Haskell program that includes your test cases and run it. What do you observe in the output?
2. Modify one of your test cases to intentionally fail. Run the program and explain how HUnit indicates that the test has failed.

\section*{Answer Key}
1. For the first problem, a simple function and test could look like this:
   ```haskell
   add :: Int -> Int -> Int
   add x y = x + y

   testAdd = TestCase (assertEqual "for (add 2 3)," 5 (add 2 3))
   ```

2. For the second problem, the function might be:
   ```haskell
   add :: Int -> Int -> Int
   add x y = x + y

   testAddNegative = TestCase (assertEqual "for (add (-2) 3)," 1 (add (-2) 3))
   testAddNegativeTwo = TestCase (assertEqual "for (add (-2) (-3))," -5 (add (-2) (-3)))
   ```

3. A sample test suite for multiplication could look like this:
   ```haskell
   multiply :: Int -> Int -> Int
   multiply x y = x * y

   testMultiply1 = TestCase (assertEqual "for (multiply 5 0)," 0 (multiply 5 0))
   testMultiply2 = TestCase (assertEqual "for (multiply 5 3)," 15 (multiply 5 3))
   testMultiply3 = TestCase (assertEqual "for (multiply 0 0)," 0 (multiply 0 0))
   ```

4. For the prime checking function:
   ```haskell
   isPrime :: Int -> Bool
   isPrime n = n > 1 && all (\x -> n `mod` x /= 0) [2..(floor . sqrt . fromIntegral) n]

   testPrimeTrue = TestCase (assertEqual "for (isPrime 5)," True (isPrime 5))
   testPrimeFalse = TestCase (assertEqual "for (isPrime 4)," False (isPrime 4))
   ```

5. To run the tests and check for failures, you can use:
   ```haskell
   main :: IO ()
   main = runTestTT tests
   ```

This worksheet is designed to help students practice and understand HUnit in Haskell. It includes explanations, practice problems, and an answer key for self-assessment.

\end{document}
```