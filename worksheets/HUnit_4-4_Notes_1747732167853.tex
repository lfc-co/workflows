```latex
\documentclass{article}
\usepackage{amsmath}
\usepackage{amsfonts}
\usepackage{amssymb}
\usepackage{enumitem}

\title{Worksheet ? HUnit\_4-4\_Notes ? HUnit\_4-4\_Notes}
\author{}
\date{}

\begin{document}
\maketitle

\section*{Topic 1: Introduction to H-Units}
H-Units are a measure of energy that indicate the amount of energy required to perform a specific task. Understanding H-Units is essential for analyzing energy consumption in various applications.

\subsection*{Explanatory Notes}
H-Units can be calculated using the formula:
\[
\text{H-Units} = \text{Power (in watts)} \times \text{Time (in hours)}
\]
This formula helps in determining the total energy used over a period.

\subsection*{Practice Problems}
1. Calculate the H-Units for a device that uses 150 watts for 3 hours.
2. If a machine operates at 200 watts for 5 hours, how many H-Units does it consume?
3. A light bulb uses 60 watts for 10 hours. What is the total energy in H-Units?

\subsection*{Answer Key}
1. H-Units = 150 W $\times$ 3 h = 450 H-Units
2. H-Units = 200 W $\times$ 5 h = 1000 H-Units
3. H-Units = 60 W $\times$ 10 h = 600 H-Units

\section*{Topic 2: Applications of H-Units}
H-Units are applicable in various fields such as household energy consumption, industrial applications, and energy efficiency assessments.

\subsection*{Explanatory Notes}
By understanding how to calculate and interpret H-Units, individuals and businesses can make informed decisions about energy usage and implement energy-saving strategies.

\subsection*{Practice Problems}
1. A factory uses machines that consume a total of 5000 watts over 8 hours. Calculate the H-Units used.
2. If a household appliance consumes 1200 watts for 2 hours, how many H-Units does it use?
3. Determine the energy usage in H-Units for a solar panel system generating 300 watts for 12 hours.

\subsection*{Answer Key}
1. H-Units = 5000 W $\times$ 8 h = 40000 H-Units
2. H-Units = 1200 W $\times$ 2 h = 2400 H-Units
3. H-Units = 300 W $\times$ 12 h = 3600 H-Units

\section*{Topic 3: Energy Efficiency and H-Units}
Understanding H-Units is crucial for improving energy efficiency in both residential and commercial settings.

\subsection*{Explanatory Notes}
By monitoring H-Units, one can identify high energy-consuming devices and seek alternatives or upgrades to more efficient models.

\subsection*{Practice Problems}
1. A refrigerator uses 250 watts for 24 hours. Calculate its H-Units.
2. If an electric heater operates at 1500 watts for 6 hours, how many H-Units does it consume?
3. A computer runs at 300 watts for 10 hours. What is its energy usage in H-Units?

\subsection*{Answer Key}
1. H-Units = 250 W $\times$ 24 h = 6000 H-Units
2. H-Units = 1500 W $\times$ 6 h = 9000 H-Units
3. H-Units = 300 W $\times$ 10 h = 3000 H-Units

\end{document}
```