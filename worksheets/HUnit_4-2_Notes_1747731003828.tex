```latex
\documentclass{article}
\usepackage{amsmath}
\usepackage{amssymb}
\usepackage{enumitem}

\title{Worksheet ? HUnit\_4-2\_Notes ? HUnit\_4-2\_Notes}
\author{}
\date{}

\begin{document}
\maketitle

\section*{Topic 1: Understanding Units of Measurement}
Units of measurement are essential for quantifying physical quantities. Common units include meters for distance, kilograms for mass, and seconds for time. Understanding the relationships between different units is crucial in solving problems involving measurements.

\subsection*{Explanatory Notes}
- **Metric System**: The metric system is the most widely used system of measurement. It is based on powers of ten, making conversions straightforward.
- **Imperial System**: The imperial system is used primarily in the United States, with units like feet, pounds, and gallons.
- **Conversion Factors**: To convert between units, you can use conversion factors (e.g., 1 inch = 2.54 cm).

\subsection*{Practice Problems}
1. Convert 5 kilometers to meters.
2. If a recipe requires 2 cups of flour, how many milliliters is that? (1 cup = 236.6 ml)
3. A car travels 60 miles per hour. What is its speed in kilometers per hour? (1 mile = 1.60934 km)

\section*{Topic 2: Area and Perimeter}
Calculating area and perimeter is fundamental in geometry. The area measures the space inside a shape, while the perimeter measures the distance around it.

\subsection*{Explanatory Notes}
- **Rectangle**: Area = length $\times$ width; Perimeter = 2(length + width)
- **Circle**: Area = $\pi r^2$; Circumference (perimeter) = $2\pi r$
- **Triangle**: Area = $\frac{1}{2} \times \text{base} \times \text{height}$; Perimeter = sum of all sides

\subsection*{Practice Problems}
1. Find the area and perimeter of a rectangle with length 8 cm and width 3 cm.
2. Calculate the area of a circle with a radius of 5 cm.
3. A triangle has sides of lengths 3 cm, 4 cm, and 5 cm. What is its perimeter?

\section*{Answer Key}
\subsection*{Topic 1: Understanding Units of Measurement}
1. \(5 \text{ km} = 5000 \text{ m}\)
2. \(2 \text{ cups} = 473.2 \text{ ml}\)
3. \(60 \text{ mph} \approx 96.56 \text{ km/h}\)

\subsection*{Topic 2: Area and Perimeter}
1. Area = \(24 \text{ cm}^2\), Perimeter = \(22 \text{ cm}\)
2. Area = \(78.54 \text{ cm}^2\)
3. Perimeter = \(12 \text{ cm}\)

\end{document}
```