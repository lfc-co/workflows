```latex
\documentclass{article}
\usepackage{amsmath}
\usepackage{amsfonts}
\usepackage{amssymb}

\title{Worksheet ? HUnit\_4-3\_Notes ? HUnit\_4-3\_Notes}
\author{}
\date{}

\begin{document}

\maketitle

\section*{Understanding Ratios}
A ratio is a relationship between two numbers indicating how many times the first number contains the second. Ratios can be written in several forms: as fractions, with a colon, or in words.

\subsection*{Explanatory Notes}
For example, the ratio of 2 to 3 can be expressed as:
\[
\frac{2}{3}, \quad 2:3, \quad \text{or } 2 \text{ to } 3.
\]
Ratios can also be simplified just like fractions. If we have the ratio 4:8, we can simplify it to:
\[
\frac{4}{8} = \frac{1}{2} \quad \Rightarrow \quad 1:2.
\]

\subsection*{Practice Problems}
1. Express the ratio of 5 to 15 in simplest form.
2. If a recipe requires 2 cups of flour and 3 cups of sugar, what is the ratio of flour to sugar?
3. Simplify the ratio 12:16.

\section*{Proportions}
A proportion is an equation that states that two ratios are equal. 

\subsection*{Explanatory Notes}
For example, if we say that the ratio of boys to girls in a class is 2:3, we can express this as a proportion:
\[
\frac{2}{3} = \frac{4}{6}.
\]
To solve proportions, we can use cross multiplication.

\subsection*{Practice Problems}
1. Solve for \(x\): \(\frac{3}{4} = \frac{x}{12}\).
2. If \(\frac{5}{x} = \frac{15}{18}\), find the value of \(x\).
3. Verify if the following ratios form a proportion: \(8:12\) and \(2:3\).

\section*{Answer Key}
\subsection*{Understanding Ratios}
1. \(\frac{5}{15} = \frac{1}{3} \quad \Rightarrow \quad 1:3\)
2. Flour to sugar ratio: \(2:3\)
3. \(12:16 = 3:4\)

\subsection*{Proportions}
1. \(3x = 48 \quad \Rightarrow \quad x = 16\)
2. Cross multiply: \(5 \cdot 18 = 15 \cdot x \quad \Rightarrow \quad 90 = 15x \quad \Rightarrow \quad x = 6\)
3. Yes, both ratios simplify to \(\frac{2}{3}\).

\end{document}
```