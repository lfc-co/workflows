```latex
\documentclass{article}
\usepackage{amsmath}
\usepackage{amsfonts}
\usepackage{amssymb}
\usepackage{geometry}
\geometry{a4paper, margin=1in}

\title{Worksheet ? HUnit\_4-4\_Notes ? HUnit\_4-4\_Notes}
\author{}
\date{}

\begin{document}

\maketitle

\section*{Understanding HUnit Testing}
HUnit is a unit testing framework for the Haskell programming language. It allows developers to write test cases for their functions to ensure that they behave as expected.

\subsection*{Explanatory Notes}
Unit testing is essential in software development as it helps catch bugs early in the development process. In HUnit, a test case is defined using assertions that compare the expected output of a function with its actual output.

\subsection*{Practice Problems}
\begin{enumerate}
    \item Write a test case for a function `add` that adds two integers. The expected output for `add(2, 3)` should be `5`.
    \item Create a test for a function `divide` that divides two numbers. Ensure to test for a division by zero case. The expected output for `divide(10, 2)` should be `5`, and `divide(10, 0)` should handle the error gracefully.
    \item Write a test case for a function `concatStrings` that concatenates two strings. The expected output for `concatStrings("Hello, ", "World!")` should be `"Hello, World!"`.
\end{enumerate}

\section*{Implementing HUnit Tests}
To implement HUnit tests, you need to import the HUnit library and create test cases using the `TestCase` and `TestList` structures.

\subsection*{Explanatory Notes}
Each test case can be run individually, and the results can be reported. You can group multiple test cases into a test suite for easier management.

\subsection*{Practice Problems}
\begin{enumerate}
    \item Import the HUnit library and create a test suite for your `add` function defined previously.
    \item Add more test cases to your suite for edge cases, such as adding negative numbers.
    \item Create a complete test suite for your `divide` function, ensuring all edge cases are covered.
\end{enumerate}

\section*{Answer Key}
\subsection*{Answers to Practice Problems - Understanding HUnit Testing}
\begin{enumerate}
    \item 
    \begin{verbatim}
    testAdd = TestCase (assertEqual "for (add 2 3)," 5 (add 2 3))
    \end{verbatim}
    \item 
    \begin{verbatim}
    testDivide = TestCase (assertEqual "for (divide 10 2)," 5 (divide 10 2))
    testDivideByZero = TestCase (assertThrows "for (divide 10 0)" (divide 10 0))
    \end{verbatim}
    \item 
    \begin{verbatim}
    testConcatStrings = TestCase (assertEqual "for (concatStrings \"Hello, \" \"World!\")," "Hello, World!" (concatStrings "Hello, " "World!"))
    \end{verbatim}
\end{enumerate}

\subsection*{Answers to Practice Problems - Implementing HUnit Tests}
\begin{enumerate}
    \item 
    \begin{verbatim}
    import Test.HUnit

    testSuite = TestList [testAdd]
    \end{verbatim}
    \item 
    \begin{verbatim}
    testAddNegative = TestCase (assertEqual "for (add -2 -3)," -5 (add -2 -3))
    \end{verbatim}
    \item 
    \begin{verbatim}
    testSuite = TestList [testDivide, testDivideByZero]
    \end{verbatim}
\end{enumerate}

\end{document}
```