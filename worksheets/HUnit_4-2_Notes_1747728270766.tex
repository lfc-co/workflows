\documentclass{article}
\usepackage[utf8]{inputenc}
\title{Worksheet ? HUnit_4-2_Notes}
\begin{document}
\maketitle
\section*{HUnit_4-2_Notes}
Your content here...\\
\end{document}