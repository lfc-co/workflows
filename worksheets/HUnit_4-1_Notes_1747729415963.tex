```latex
\documentclass{article}
\usepackage[utf8]{inputenc}
\usepackage{amsmath}
\usepackage{amssymb}

\title{Worksheet ? HUnit_4-1_Notes ? HUnit_4-1_Notes}
\author{}
\date{}

\begin{document}

\maketitle

\section*{Introduction to HUnit}
HUnit is a unit testing framework for Haskell. It allows developers to write tests for their functions and ensures that the code behaves as expected. In this section, we will cover the basics of HUnit, including how to set up tests and the structure of test cases.

\subsection*{Explanatory Notes}
Unit testing is a crucial part of software development. HUnit provides a simple way to write tests in Haskell. Tests are written as functions that can assert whether the results of code match expected outcomes.

\subsection*{Practice Problems}
1. Write a simple function in Haskell that adds two numbers. Create a test case using HUnit to check if your function works correctly for the input values 2 and 3.

2. Modify the function from problem 1 to handle cases where one of the inputs is negative. Write a test case to validate this functionality.

\section*{Writing Test Cases}
In HUnit, test cases are typically written using the `TestCase` constructor. Each test case can make assertions using functions like `assertEqual`.

\subsection*{Explanatory Notes}
A test case consists of a description and an assertion. For example, `assertEqual` checks if the expected result matches the actual result.

\subsection*{Practice Problems}
1. Create a test case for a function that multiplies two numbers. Ensure that your test checks both positive and negative numbers.

2. Write a test case that checks for the correct handling of zero in your multiplication function.

\section*{Running Tests}
To run the tests you have written, you will need to use the HUnit framework's test runner.

\subsection*{Explanatory Notes}
The test runner will execute all tests and report results, showing which tests passed and which failed, along with any relevant error messages.

\subsection*{Practice Problems}
1. Set up a simple test runner for your tests. Make sure it outputs the results clearly.

2. Modify your test cases to include more edge cases and re-run your tests to see if they pass.

\section*{Answer Key}
1. The function that adds two numbers should return their sum. The test case should use `assertEqual` to compare the result to 5 when the inputs are 2 and 3.

2. The modified function should handle negative numbers correctly. The test case should check the result for inputs like -1 and 1.

3. For the multiplication function, the test should check that multiplying 2 with 3 returns 6 and that multiplying -2 with 3 returns -6.

4. In the test case for zero, multiplying any number by zero should return zero.

5. Your test runner should call `runTestTT` and print the results to the console.

\end{document}
```