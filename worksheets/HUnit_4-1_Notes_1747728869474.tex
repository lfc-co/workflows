```latex
\documentclass{article}
\usepackage{amsmath}
\usepackage{amssymb}

\title{Worksheet ? HUnit\_4-1\_Notes ? HUnit\_4-1\_Notes}
\author{}
\date{}

\begin{document}

\maketitle

\section*{Topic 1: Introduction to HUnit}
HUnit is a unit testing framework for Haskell that allows developers to test their code effectively. It provides a way to write test cases and run them to ensure that the code behaves as expected. Being familiar with HUnit will help you in writing better, more reliable Haskell applications.

\subsection*{Explanatory Notes}
- HUnit allows you to define test cases as functions.
- Each test case can check for expected outcomes using assertions.
- You can group test cases into test suites for organized testing.

\subsection*{Practice Problems}
1. Write a simple HUnit test case that checks if the function `add` correctly adds two numbers.
2. Create a test suite that includes at least three test cases for different functions.

\subsection*{Answer Key}
1. 
\begin{verbatim}
testAdd = TestCase (assertEqual "for (add 2 3)," 5 (add 2 3))
\end{verbatim}
2. 
\begin{verbatim}
tests = TestList [testAdd, testSubtract, testMultiply]
\end{verbatim}

\section*{Topic 2: Writing Test Cases}
Writing effective test cases is crucial for validating the functionality of your code. Each test case should be specific, focused on a single behavior of a function.

\subsection*{Explanatory Notes}
- A test case should describe what it tests and what the expected outcome is.
- Use descriptive names for your test cases to make it easier to understand what is being tested.

\subsection*{Practice Problems}
1. Define a test case for a function that calculates the factorial of a number.
2. Write a test case to check if the function `divide` throws an error when dividing by zero.

\subsection*{Answer Key}
1. 
\begin{verbatim}
testFactorial = TestCase (assertEqual "for (factorial 5)," 120 (factorial 5))
\end{verbatim}
2. 
\begin{verbatim}
testDivideByZero = TestCase (assertThrows "divide by zero" (divide 5 0))
\end{verbatim}

\section*{Topic 3: Running Tests}
Once you have written your test cases, you need to run them to see if your code passes all the tests. 

\subsection*{Explanatory Notes}
- You can run your tests using the `runTestTT` function provided by HUnit.
- It's important to regularly run your tests as you develop your code.

\subsection*{Practice Problems}
1. Write a Haskell script that includes your test suite and runs it.
2. Explain what the output of the test runner indicates.

\subsection*{Answer Key}
1. 
\begin{verbatim}
main = runTestTT tests
\end{verbatim}
2. The output indicates which tests passed and which failed, along with any assertion messages.

\end{document}
```