```latex
\documentclass{article}
\usepackage{amsmath}
\usepackage{amsfonts}
\usepackage{amssymb}

\title{Worksheet ? HUnit_4-3_Notes ? HUnit_4-3_Notes}
\author{}
\date{}

\begin{document}

\maketitle

\section*{Topic 1: Understanding H-Units}
H-Units are a measure of height used in various fields including construction and architecture. They provide a standardized way to describe vertical dimensions.

\textbf{Explanatory Notes:}  
An H-Unit is defined as a specific measurement that can be converted into other units of height such as meters or feet. Understanding H-Units is crucial for accurate planning and design.

\textbf{Practice Problems:}  
1. Convert 10 H-Units to meters if 1 H-Unit = 0.3048 meters.  
2. If a building is 25 H-Units tall, how many feet is that? (1 H-Unit = 3.28084 feet)

\section*{Topic 2: Applications of H-Units}
H-Units are extensively used in blueprints and construction plans to ensure that all measurements adhere to regulatory standards.

\textbf{Explanatory Notes:}  
When reading blueprints, knowing how to interpret H-Units can help in understanding the scale and proportion of structures. This is critical in ensuring that buildings are built to accurate specifications.

\textbf{Practice Problems:}  
1. A blueprint indicates a wall height of 15 H-Units. If each H-Unit is 2.5 meters, what is the actual height of the wall in meters?  
2. If a room is 12 H-Units wide and 10 H-Units long, what is the area of the room in square H-Units?

\section*{Topic 3: Conversion and Calculations with H-Units}
Knowing how to convert H-Units to other units can help in practical applications and ensure precision in measurements.

\textbf{Explanatory Notes:}  
Conversions often require using conversion factors. Familiarizing oneself with these factors allows for quick and efficient calculations.

\textbf{Practice Problems:}  
1. Convert 5 H-Units to inches. (1 H-Unit = 39.3701 inches)  
2. If a structure is 50 H-Units tall, how many centimeters is that? (1 H-Unit = 100 centimeters)

\section*{Answer Key}
\textbf{Topic 1:}  
1. \(10 \text{ H-Units} \times 0.3048 \text{ m/H-Unit} = 3.048 \text{ m}\)  
2. \(25 \text{ H-Units} \times 3.28084 \text{ ft/H-Unit} = 82.020 ft\)

\textbf{Topic 2:}  
1. \(15 \text{ H-Units} \times 2.5 \text{ m/H-Unit} = 37.5 \text{ m}\)  
2. Area = \(12 \text{ H-Units} \times 10 \text{ H-Units} = 120 \text{ square H-Units}\)

\textbf{Topic 3:}  
1. \(5 \text{ H-Units} \times 39.3701 \text{ in/H-Unit} = 196.8505 \text{ in}\)  
2. \(50 \text{ H-Units} \times 100 \text{ cm/H-Unit} = 5000 \text{ cm}\)

\end{document}
```