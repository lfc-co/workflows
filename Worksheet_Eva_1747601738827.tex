\documentclass[12pt]{article}
\usepackage[margin=0.5in]{geometry}
\usepackage{amsmath}
\usepackage{enumerate}

\begin{document}

\title{Chemistry Worksheet: Stoichiometry and Yield}
\author{For Eva}
\date{}
\maketitle

\section*{Introduction to Stoichiometry}
Stoichiometry is the part of chemistry that involves calculating the quantities of reactants and products in chemical reactions. Understanding stoichiometry allows us to predict how much product can be generated from given amounts of reactants, and it is fundamental for various applications, including cooking and industrial manufacturing.

\section*{Learning Objectives}
By the end of this worksheet, you will:
\begin{itemize}
    \item Understand the concept of limiting and excess reagents in chemical reactions.
    \item Calculate percent yield in chemical reactions.
    \item Differentiate between empirical and molecular formulas.
    \item Solve for empirical and molecular formulas based on given information.
\end{itemize}

\section*{Instructions}
Complete the guided notes below and then work through the practice problems. Show all your work for full credit.

\subsection*{Guided Notes}

\textbf{1. Limiting and Excess Reagents:}
\begin{itemize}
    \item The \textbf{limiting reagent} is the reactant that is completely consumed in a chemical reaction, limiting the amount of product formed.
    \item The \textbf{excess reagent} is the reactant that is not completely used up.
\end{itemize}

\textbf{Example:} For the reaction \( 2H_2 + O_2 \rightarrow 2H_2O \):
\begin{itemize}
    \item If you start with 4 moles of \( H_2 \) and 2 moles of \( O_2 \), \( O_2 \) is the limiting reagent because it will be consumed first.
\end{itemize}

\textbf{2. Percent Yield:}
\begin{itemize}
    \item \textbf{Percent yield} is calculated using the formula:
    \[
    \text{Percent Yield} = \left( \frac{\text{Actual Yield}}{\text{Theoretical Yield}} \right) \times 100
    \]
\end{itemize}

\textbf{Example:} If the theoretical yield of water is 10 g, and you actually produced 8 g:
\[
\text{Percent Yield} = \left( \frac{8}{10} \right) \times 100 = 80\%
\]

\textbf{3. Empirical vs. Molecular Formula:}
\begin{itemize}
    \item The \textbf{empirical formula} is the simplest whole number ratio of elements in a compound.
    \item The \textbf{molecular formula} shows the actual number of atoms of each element in a molecule.
\end{itemize}

\textbf{Example:} For glucose, \( C_6H_{12}O_6 \):
\begin{itemize}
    \item Empirical formula: \( CH_2O \) (ratio 1:2:1)
\end{itemize}

\subsection*{Practice Problems}

\textbf{Basic Problems}
\begin{enumerate}[1.]
    \item In a reaction between 3 moles of \( A \) and 5 moles of \( B \), which is the limiting reagent if the reaction ratio is \( 1:2 \) (A:B)? 
    \textit{(Hint: Determine how many moles of \( B \) are needed for \( A \) to react.)}
    
    \item Calculate the percent yield if the theoretical yield of a product is 15 g, and the actual yield is 12 g.
\end{enumerate}

\textbf{Intermediate Problems}
\begin{enumerate}[3.]
    \item Determine the empirical formula of a compound containing 40\% Carbon, 6.7\% Hydrogen, and 53.3\% Oxygen by mass. 
    \textit{(Hint: Convert percentages to moles.)}
    
    \item A sample of a compound has an empirical formula of \( CH_2 \) and a molar mass of 28 g/mol. What is its molecular formula? 
    \textit{(Hint: Divide the molar mass by the mass of the empirical formula.)}
\end{enumerate}

\textbf{Advanced Problem}
\begin{enumerate}[5.]
    \item Given the reaction \( 2C + 2H_2 \rightarrow C_2H_4 \), if you start with 4 moles of \( C \) and 3 moles of \( H_2 \), what is the limiting reagent and how many moles of \( C_2H_4 \) can be produced?
\end{enumerate}

\subsection*{Answer Key}

\begin{enumerate}[1.]
    \item Limiting reagent is \( A \) (requires 2 moles of \( B \) for every mole of \( A \)).
    \item Percent Yield = \( \left( \frac{12}{15} \right) \times 100 = 80\% \).
    \item Empirical formula = \( C_2H_4O \).
    \item Molecular formula = \( C_4H_8 \).
    \item Limiting reagent is \( H_2 \); 2 moles of \( C_2H_4 \) can be produced.
\end{enumerate}

\subsection*{Common Mistakes to Avoid}
\begin{itemize}
    \item Miscalculating moles from grams or percentages.
    \item Forgetting to convert units when necessary.
    \item Confusing limiting and excess reagents.
\end{itemize}

\subsection*{Glossary}
\begin{itemize}
    \item \textbf{Limiting Reagent:} The reactant that is fully consumed in a reaction.
    \item \textbf{Excess Reagent:} The reactant that is not fully consumed in a reaction.
    \item \textbf{Theoretical Yield:} The maximum amount of product that can be formed from given amounts of reactants.
    \item \textbf{Actual Yield:} The amount of product actually obtained from a reaction.
    \item \textbf{Empirical Formula:} The simplest whole number ratio of elements in a compound.
    \item \textbf{Molecular Formula:} The actual number of atoms of each element in a molecule.
\end{itemize}

\subsection*{Self-Reflection}
After completing this worksheet, take a moment to reflect on what you learned. Write down one thing you found easy and one thing you found challenging.

\end{document}